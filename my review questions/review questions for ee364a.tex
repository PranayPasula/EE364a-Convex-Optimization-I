\documentclass[]{article}
\usepackage[margin=1.25in]{geometry}
\usepackage{amssymb}
\usepackage{amsmath}
\usepackage{calc}
\usepackage{enumitem}
\setlist[enumerate]{itemsep=-1mm, leftmargin=!}

\newcommand{\R}{\mathbb{R}}
\newcommand{\Rn}{\mathbb{R}^{n}}
\newcommand{\Rnp}{\boldsymbol{R}^{n}_{+}}
\newcommand{\Rpp}{\boldsymbol{R}_{++}}
\newcommand{\Rnpp}{\boldsymbol{R}^{n}_{++}}
\newcommand{\Rmn}{\mathbb{R}^{m\text{x}n}}
\newcommand{\norm}[1]{\| #1 \|}
\newcommand{\Sn}{\boldsymbol{S}^{n}}
\newcommand{\Snp}{\boldsymbol{S}^{n}_{+}}
\newcommand{\Snpp}{\boldsymbol{S}^{n}_{++}}

\newcounter{q}
\newcommand{\qnum}{\addtocounter{q}{1}\theq.\quad}

%opening
\title{Review Questions for EE364a -- Convex Optimization I}
\author{Pranay Pasula}

\begin{document}
\maketitle

\section*{Intro}
From my experience, asking yourself questions and answering them is one of the best ways to test how well you understand something. 
\\\\
Most of the questions here are simple and test for only a basic understanding. Some questions require some insight, but almost none of them should be "challenging."
\\
\section*{Chapter 2 -- Convex Sets}

\subsubsection*{Affine Set}

\begin{enumerate}
 \item What is an affine set?
 \item Describe an affine set using set notation.
 \item What are some examples of affine sets?
 \item Is any affine set the solution to a linear system of equations?
\end{enumerate}

\subsubsection*{Convex Set}

\begin{enumerate}[resume*]
\item What is a convex set? 
\item Describe a convex set using set notation. 
\item Are affine sets also convex sets? Why or why not? 
\item Are null sets also convex sets? Why or why not? 
\item What do convex sets look like in $\Rn$? 
\item How does the definition of convexity apply to a set $S \nsubseteq \Rn$? Say, for $S = \Snpp$? 
\item What is the interior of a set? 
\item Give examples of convex sets with an interior equal to the empty set.
\end{enumerate}

\subsubsection*{Convex Combination}

\begin{enumerate}[resume*]
\item What is a convex combination? 
\item What is a convex hull? 
\item What is the convex hull of a convex set? 
\item How do convex combinations differ from affine combinations?
\end{enumerate}

\subsubsection*{Convex Cone}

\begin{enumerate}[resume*]
\item What is a cone? 
\item What is a convex cone? 
\item Give an example of a cone that is not convex.
\item Why is a convex cone considered to be convex even though the constraints on weights are different than what is required by the definition of convexity? 
\item What is a conic combination? 
\item What is a conic hull? 
\item What convex cone that contains a set $S$ is bigger than the conic hull of $S$?
\end{enumerate}

\subsubsection*{Hyperplane}

\begin{enumerate}[resume*]
\item What is a hyperplane? 
\item Describe a hyperplane using set notation. 
\item Describe how the variables in the definition of a hyperplane determine the hyperplane. 
\item Is a hyperplane a vector space? 
\item Is a hyperplane a subspace? 
\item Is a hyperplane an affine subspace? 
\item How does the definition of an affine subspace differ from the definition of a vector space? How do the relate? 
\item Does every affine subspace contain a vector space? 
\item Does every vector space contain an affine subspace? 
\item What is the dimension of a hyperplane? 
\item What is the codimension of a hyperplane? 
\item What is meant by codimension in this context? When does is the codimension defined in this context? 
\item Is a hyperplane convex? Affine? 
\end{enumerate}

\subsubsection*{Halfspace}

\begin{enumerate}[resume*]
\item What is a halfspace?
\item If a halfspace contains the origin, is it a vector space?
\item Is a halfspace convex?
\item Is a halfspace affine?
\item How does the definition of a halfspace relate to the definition of a hyperplane?
\end{enumerate}

\subsubsection*{Euclidean Ball and Ellipsoid}

\begin{enumerate}[resume*]
\item What is a euclidean ball?
\item Give two representations of a euclidean ball using set notation.
\item What is an ellipsoid?
\item Give two representations of an ellipsoid using set notation.
\item If P (in the first representation) is positive definite, what does this imply about A (in the second representation)?
\item What are the lengths of the semi-major axes of the ellipsoid equal to?
\item What if P is not positive definite?
\item What is the affine dimension of such an ellipsoid? \textbf{I'm unsure why Boyd's book makes the claim it does regarding this. It doesn't seem like a degenerate ellipsoid has an affine dimension because the there doesn't exist a translation that makes the set a vector space.}
\item Which of the following are not convex: euclidean ball, non-degenerate ellipsoid, degenerate ellipsoid?
\end{enumerate}

\subsubsection*{Norm Ball and Norm Cone}

\begin{enumerate}[resume*]
\item What is a norm?
\item What is a p-norm?
\item Describe the absolute value function as a norm.
\item What is a norm ball?
\item What is a norm cone?
\item How do the norm balls of a particular norm relate to the norm cone of the same norm?
\item What is the unit norm ball?
\item What is the $p$-"norm?"
\item What does the unit $p$-"norm" ball look like for $0 < p < 1$? For $p=0?$
\item Are norm balls convex?
\item Are norm cones convex?
\item Why are $p$-"norms" with $0\leq p < 1$ not true norms?
\item When are such norms useful?
\end{enumerate}

\subsubsection*{Polyhedra}

\begin{enumerate}[resume*]
\item What is a polyhedra in set notation? What does this imply about polyhedra in terms of other convex sets?
\item Are rays polyhedra? Are line segments?
\item Are polyhedra also convex?
\item What is a bounded polyhedron sometimes called?
\item What is the nonnegative orthant? Is it a polyhedron? Is it a cone?
\item What is the positive orthant? Is it a polyhedron? Is it a cone?
\item What are polyhedrons that are also cones sometimes called?
\end{enumerate}

\subsubsection*{Positive Semidefinite Cone}
	
\begin{enumerate}[resume*]
\item What is the positive semidefinite cone?
\item Is $\Snp$ convex? Affine? Linear?
\item Is $\Snpp$ also a convex cone?
\end{enumerate}

\subsubsection*{Operations that Preserve Convexity}

\begin{enumerate}[resume*]
\item What are two ways to show that a set is convex?
\item What is an easy way to show that a complicated set is not convex?
\item What are pros and cons of each of these approaches?
\item Name four convexity-preserving types of operations.
\item Explicity describe each of these operations.
\item Is the inverse image of a convex set under f also convex? Is the converse true?
\item State examples of types of operations that can be shown to be convex through affine functions.
\item What is a linear matrix inequality? Is its solution set convex?
\item What is the perspective function on a vector?
\item What is the generalization of the perspective function called? Describe it.
\end{enumerate}

\subsubsection*{Generalized Inequalities}

\begin{enumerate}[resume*]
\item What is a proper cone?
\item Give examples of sets that are proper cones.
\item Give examples of sets that are cones but not proper cones.
\item Are all convex cones also proper cones?
\item Are all proper cones convex?
\item Is a line in $\Rn$ a proper cone? A ray?
\item Is a vector space a proper cone? An affine subspace?
\item Is a hyperplane a proper cone? A halfspace? A polyhedron?
\item Is $\Sn$ a proper cone? $\Snp$? $\Snpp$?
\item Is $\Rn$ a proper cone? $\Rnp$? $\Rnpp$?
\item What is a generalized inequality?
\item What does the generalized inequality imply when $K=\Rnp$? When $K=\Snp$?
\item What is the interior of $\Snp$ called?
\item Name a property of $\preceq_{K}$ that has an analog on $\R$.
\item Name a property of $\preceq_{K}$ that does not have an analog on $\R$.
\end{enumerate}

\subsubsection*{Minimum and Minimal Elements}

\begin{enumerate}[resume*]
\item What is a minimum element of a set wrt a generalized inequality? Is it unique? Does one always exist?
\item What is a minimal element of a set wrt a generalized inequality? Is it unique? Does one always exist?
\item When does every minimum equal every minimal element?
\item In what region is the ordering wrt a proper cone $K$ unambiguous? In what region is it ambiguous?
\item If all other points in a set are more than a point $x$ by $\preceq_{K}$, then is $x$ a minimum element? A minimal element? Both?
\item If no other point in a set is less than a point $x$ by $\preceq_{K}$, then is $x$ a minimum element? A minimal element? Both?
\end{enumerate}

\subsection*{Separating Hyperplane Theorem}

\begin{enumerate}[resume*]
\item State the separating hyperplane theorem.
\item What is required for strict separation?
\item Give an example of two disjoint convex sets for which there does not exist a strict separating hyperplane.
\end{enumerate}

\subsection*{Supporting Hyperplane Theorem}

\begin{enumerate}[resume*]
\item What is a supporting hyperplane?
\item Can nonconvex sets have a supporting hyperplane?
\item What sets are separated by a supporting hyperplane?
\item Geometrically, in Euclidean space, what is the supporting hyperplane at a particular point?
\item State the supporting hyperplane theorem.
\item Do all convex sets have at least one supporting hyperplane?
\end{enumerate}

\subsection*{Dual Cones and Generalized Inequalities}

\begin{enumerate}[resume*]
\item What is a dual cone?
\item What is the maximum angle between any vector in a cone and any vector in its corresponding dual? Minimum angle?
\item How does the dual cone relate to supporting hyperplanes?
\item When is a dual cone nonconvex?
\item When is a cone nonconvex?
\item What is the dual cone of a vector space? Why is the dual cone of a vector space defined?
\item What are some examples of cones and their duals?
\item What are some examples of self-dual cones?
\item What are some examples of cones that are not self-dual?
\item Is $K^{*}$ always closed?
\item If $K_{1} \subseteq K_{2}$, then what is the relationship between $K_{1}^{*}$ and $K_{2}^{*}$?
\item Is $K^{*}$ always pointed?
\item When does the $(K^{*})^{*}=K$? Otherwise, what does $(K^{*})^{*}$ equal?
\item If K is a proper cone, then is $K^{*}$ always a proper cone? If not, give a counterexample.
\end{enumerate}

\subsubsection*{Dual Generalized Inequality}

\begin{enumerate}[resume*]
\item What is the dual of a generalized inequality?
\item What is a property relating a generalized inequality and its dual?
\item How does this property change when we set $K=K^{*}$?
\end{enumerate}

\subsubsection*{Minimum and Minimal Elements via Dual Inequalities}

\begin{enumerate}[resume*]
\item State the dual characterization of the minimum element of a set wrt a generalized inequality.
\item How does this relate to supporting hyperplanes?
\item Does the set have to be convex? If not, give an example of a nonconvex set for which the dual characterization of the minimum element holds.
\item State the dual characterization of minimal elements of a set wrt a generalized inequality.
\item How does this relate to supporting hyperplanes?
\item Does the set have to be convex? If not, give an example of a nonconvex set for which the dual characterization of the minimum element holds.
\item If the set is convex, what is implied about the minimal elements of the set (note: minimal, not minimal wrt a generalized equality)?
\end{enumerate}

\subsubsection*{Optimal Production Frontier}

\begin{enumerate}[resume*]
\item Describe the production--possibility frontier.
\item Describe its regions.
\item What solutions are considered efficient?
\item What is another name for an efficient solution?\\
\end{enumerate}

\section*{Chapter 3 -- Convex Functions}

\subsubsection*{Definition}
\begin{enumerate}
\item What is the definition of a convex function?
\item How is this extended to an arbitrary number of variables?
\item Graphically, what is the definition of a convex function?
\item What is the definition of a concave function?
\item What is the definition of a strictly convex function?
\end{enumerate}

\subsubsection*{Examples on $\R$}
\begin{enumerate}[resume*]
\item State whether the following is convex, concave, or both for particular ranges of variables on non-empty subset(s) of $\R$, and state these ranges and subset(s):
\begin{itemize}
	\item Affine: $ax+b$
	\item Exponential: $e^{ax}$
	\item Power: $x^{a}$
	\item Power of abs val: $|x|^{a}$
	\item Negative entropy: $x\:$log$\:x$
	\item Logarithm: log$\:x$
\end{itemize}
\item What can be said about a function if it is both convex and concave? Explain why.
\end{enumerate}

\subsubsection*{Examples on $\Rn$ and $\Rmn$}
\begin{enumerate}[resume*]
\item Are all norms convex? Show why or why not.
\item What is the definition of a $p$-norm? For what values of $p$ is this valid? What are some uses of $p$-"norms" with invalid values of $p$?
\item What is the general form of an affine function on $\Rmn$? What is the non-constant term? What is special about this term? What does it equal?
\item What is the spectral norm of a matrix? For what matrices is this function not convex?
\end{enumerate}

\subsubsection*{Restriction of a Convex Function to a Line}
\begin{enumerate}[resume*]
\item State the theorem relating the convexity of a function $f: \Rn \rightarrow \R$ to the convexity of $f$ when $f$ is restricted to an arbitrary line passing through $\textbf{dom}\: f$.
\item (\textit{Note: this requires work}) For $f: \Snpp \rightarrow \R$, is $\text{log}\:\text{det}\:X$ convex, concave, both, or neither?.
\item If you have no clue whether a function is convex or concave, what is usually the best thing to do first?
\end{enumerate}
\textit{Note: Restriction to a line is one of the best ways to show the convexity of a function.}

\subsubsection*{Extended-Value Extension}
\begin{enumerate}[resume*]
\item What is the extended-value extension for convex functions?
\item Why is it useful?
\item What does $\tilde{f}(\theta x + (1-\theta) y) \leq \theta \tilde{f}(x) + (1-\theta)\tilde{f}(y)$ imply? 
\item What is the extended-value extension for concave functions?
\end{enumerate}

\subsubsection*{First-Order Condition}
\begin{enumerate}[resume*]
\item What is the first-order condition for convexity?
\item How does the first-order approximation of $f$ relate to $f$?
\item What does this imply about the relationship between the gradient evaluated at any point and the global bound(s) of $f$ (consider when $f$ is convex and when $f$ is strictly convex)?
\item What does this imply about the number and types of global bound(s) (consider when $f$ is convex and when $f$ is strictly convex)?
\end{enumerate}

\subsubsection*{Second-Order Condition}
\begin{enumerate}[resume*]
\item What is the second-order condition for convexity?
\item What is a sufficient condition wrt the Hessian of a function $f$ for strict convexity? Is this also a necessary condition? Show or give counterexample.
\end{enumerate}




\end{document}
